%%
% The BIThesis Template for Bachelor Graduation Thesis
%
% 北京理工大学毕业设计(论文)第一章节 —— 使用 XeLaTeX 编译
%
% Copyright 2020 Spencer Woo
%
% This work may be distributed and/or modified under the
% conditions of the LaTeX Project Public License, either version 1.3
% of this license or (at your option) any later version.
% The latest version of this license is in
%   http://www.latex-project.org/lppl.txt
% and version 1.3 or later is part of all distributions of LaTeX
% version 2005/12/01 or later.
%
% This work has the LPPL maintenance status `maintained'.
%
% The Current Maintainer of this work is Spencer Woo.
%
% 第一章节

\chapter{前言}
\section{物理科普}
在纯粹天体物理的研究方面,相对论宇宙的光线追踪方程几十年前就已经完善了。但是缺乏针对人视觉的高质量的渲染图,导致黑洞的视觉效果还停留在科学家的草稿里,很多人对黑洞的并不了解,虽然黑洞是一个非常简单的东西,只有寥寥几个属性。

\section{影视特效}
在图形学方面,影视特效(离线渲染)已经开始了非平直时空的光线追踪渲染。主要研究方向就是在已有的物理模型基础上构建人眼所应看到的影像。

2009年的电影《星际迷航》中描绘的虫洞是一个扁平的圆环,而黑洞没有任何效果,直接消失得无影无踪。2014年的电影《星际穿越》第一次将球形的黑洞(视界)搬上荧幕,从那时候起科幻电影也开始重视真实的描述黑洞的物理特征。2017年的科幻剧《星际迷航:发现号》也开始使用球状的黑洞。当很多人通过影视作品了解到黑洞真正的样子后,就不能再用老旧刻板的印象来刻画黑洞了。

\section{现有研究}
离线渲染方面,一向追求真实的电影特效刚刚开始接受非平直时空的渲染。已经存在一些生成黑洞影像的开源渲染器,但这些渲染器存在一些诸如精度不高难以体现多重爱因斯坦环,或者使用近似的公式与实际的光线轨迹有较大差别等问题。

本文主要侧重于较高的渲染精度,来体现比较重要的几个黑洞特征。
