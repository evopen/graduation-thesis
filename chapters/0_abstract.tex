%%
% The BIThesis Template for Bachelor Graduation Thesis
%
% 北京理工大学毕业设计(论文)中英文摘要 —— 使用 XeLaTeX 编译
%
% Copyright 2020 Spencer Woo
%
% This work may be distributed and/or modified under the
% conditions of the LaTeX Project Public License, either version 1.3
% of this license or (at your option) any later version.
% The latest version of this license is in
%   http://www.latex-project.org/lppl.txt
% and version 1.3 or later is part of all distributions of LaTeX
% version 2005/12/01 or later.
%
% This work has the LPPL maintenance status `maintained'.
%
% The Current Maintainer of this work is Spencer Woo.

% 中英文摘要章节
\topskip=0pt
\zihao{-4}

\vspace*{-7mm}

\begin{center}
  \heiti\zihao{-2}\textbf{\thesisTitle}
\end{center}

\vspace*{2mm}

\addcontentsline{toc}{chapter}{摘~~~~要}
{\let\clearpage\relax \chapter*{\textmd{摘~~~~要}}}
\setcounter{page}{1}

\vspace*{1mm}

\setstretch{1.53}
\setlength{\parskip}{0em}

% 中文摘要正文从这里开始
在纯粹天体物理的研究方面,相对论宇宙的光线追踪方程几十年前就已经完善了。但是缺乏针对人视觉的高质量的渲染图,导致黑洞的视觉效果还停留在科学家的草稿里。在图形学方面,影视特效(离线渲染)已经开始了非平直时空的光线追踪渲染。主要研究方向就是在已有的物理模型基础上构建人眼所应看到的影像。

2009年的电影《星际迷航》中描绘的黑洞是一个扁平的圆环,
2014年的电影《星际穿越》第一次将球形的黑洞(视界)搬上
荧幕,从那时候起影视特效开始认识到平直空间不能用于描绘
大尺度宇宙,2017年的科幻剧《星际迷航:发现号》也开始使
用球状的黑洞。
当很多人通过影视作品了解到黑洞真正的样子后,就不能再用
老旧刻板的印象来刻画黑洞了。离线渲染方面,一向追求真实
的电影特效刚刚开始接受非平直时空的渲染,还没有完善的基
于相对论的离线渲染解决方案。实时渲染方面,英伟达最新推
出了RTX系列显卡专门在芯片上分配了一大片区域专门用于光
线追踪的计算。光线追踪专用核心在消费级显卡上还是首次出
现。这意味着实时渲染方法将从有三十年历史的光栅化
(rasterization)向路径追踪(path-tracing)转变,也就是
平直时空的光线追踪。实时渲染在非平直时空的光线追踪才还
没有起步。

\textcolor{blue}{摘要正文选用模板中的样式所定义的“正文”,每段落首行缩进 2 个字符;或者手动设置成每段落首行缩进 2 个汉字,字体:宋体,字号:小四,行距:固定值 22 磅,间距:段前、段后均为 0 行。阅后删除此段。}

\textcolor{blue}{摘要是一篇具有独立性和完整性的短文,应概括而扼要地反映出本论文的主要内容。包括研究目的、研究方法、研究结果和结论等,特别要突出研究结果和结论。中文摘要力求语言精炼准确,本科生毕业设计(论文)摘要建议 300-500 字。摘要中不可出现参考文献、图、表、化学结构式、非公知公用的符号和术语。英文摘要与中文摘要的内容应一致。阅后删除此段。}

\vspace{4ex}\noindent\textbf{\heiti 关键词:本科生,毕业设计(论文),光线追踪,相对论,黑洞,测地线}
\newpage

% 英文摘要章节
\topskip=0pt

\vspace*{2mm}

\begin{spacing}{0.95}
  \centering
  \heiti\zihao{3}\textbf{\thesisTitleEN}
\end{spacing}

\vspace*{17mm}

\addcontentsline{toc}{chapter}{Abstract}
{\let\clearpage\relax \chapter*{
  \zihao{-3}\textmd{Abstract}\vskip -3bp}}
\setcounter{page}{2}

\setstretch{1.53}
\setlength{\parskip}{0em}

% 英文摘要正文从这里开始
In order to study……

\textcolor{blue}{Abstract 正文设置成每段落首行缩进 2 字符,字体:Times New Roman,字号:小四,行距:固定值 22 磅,间距:段前、段后均为 0 行。阅后删除此段。}

\vspace{3ex}\noindent\textbf{Undergraduate, Graduation Project (Thesis), Ray Tracing, Relativity, Blackhole, Geodesic}
\newpage
