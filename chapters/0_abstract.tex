%%
% The BIThesis Template for Bachelor Graduation Thesis
%
% 北京理工大学毕业设计(论文)中英文摘要 —— 使用 XeLaTeX 编译
%
% Copyright 2020 Spencer Woo
%
% This work may be distributed and/or modified under the
% conditions of the LaTeX Project Public License, either version 1.3
% of this license or (at your option) any later version.
% The latest version of this license is in
%   http://www.latex-project.org/lppl.txt
% and version 1.3 or later is part of all distributions of LaTeX
% version 2005/12/01 or later.
%
% This work has the LPPL maintenance status `maintained'.
%
% The Current Maintainer of this work is Spencer Woo.

% 中英文摘要章节
\topskip=0pt
\zihao{-4}

\vspace*{-7mm}

\begin{center}
  \songti\zihao{-3}\textbf{\thesisTitle}
\end{center}

\vspace*{2mm}

\addcontentsline{toc}{chapter}{摘~~~~要}
{\let\clearpage\relax \chapter*{\textmd{摘~~~~要}}}
\setcounter{page}{1}

\vspace*{1mm}

\setstretch{1.53}
\setlength{\parskip}{0em}

% 中文摘要正文从这里开始
广义相对论是20世纪最伟大的理论之一,它描述了有质量的物体可以扭曲时空。要具象化时空的弯曲,可以通过观测光线的路径来间接反映时空的扭曲程度。

本文利用广义相对论推导出的史瓦西黑洞光子测地线方程,在史瓦西黑洞周围进行光线追踪,生成易于理解的黑洞图像,探讨黑洞周围光线扭曲的特点。本设计实现了一个CPU光线追踪渲染器,并尝试了GPU渲染与实时渲染,配合输入参数生成对应的中心天体周围的光线扭曲图像。

\vspace{4ex}\noindent\textbf{\heiti 关键词:光线追踪;相对论;黑洞;测地线}
\newpage

% 英文摘要章节
\topskip=0pt

\vspace*{2mm}

\begin{spacing}{0.95}
  \centering
  \heiti\zihao{3}\textbf{\thesisTitleEN}
\end{spacing}

\vspace*{17mm}

\addcontentsline{toc}{chapter}{Abstract}
{\let\clearpage\relax \chapter*{
  \zihao{-3}\textmd{Abstract}\vskip -3bp}}
\setcounter{page}{2}

\setstretch{1.53}
\setlength{\parskip}{0em}

% 英文摘要正文从这里开始
General relativity is the greatest theory of 20th century, it described that a massive object can bend spacetime. One can visualize a curved spacetime by observing path taken by light near the central body.

This article utilize the null geodesic equation that derived from general relativity to raytrace light ray near a blackhole, and generate easy-to-understand blackhole image, and summarize some characteristics of the bending of light. This implementation is not only for blackhole but also any non-rotating non-charged celestial bodies, like the sun. This design implement a CPU raytracer, and tried to use GPU for rendering that generate blackhole image based on input parameters.


\vspace{3ex}\noindent\textbf{Raytracing; Relativity; Blackhole; Geodesic}
\newpage
