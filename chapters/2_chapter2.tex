%%
% The BIThesis Template for Bachelor Graduation Thesis
%
% 北京理工大学毕业设计(论文)第一章节 —— 使用 XeLaTeX 编译
%
% Copyright 2020 Spencer Woo
%
% This work may be distributed and/or modified under the
% conditions of the LaTeX Project Public License, either version 1.3
% of this license or (at your option) any later version.
% The latest version of this license is in
%   http://www.latex-project.org/lppl.txt
% and version 1.3 or later is part of all distributions of LaTeX
% version 2005/12/01 or later.
%
% This work has the LPPL maintenance status `maintained'.
%
% The Current Maintainer of this work is Spencer Woo.
%
% 第一章节

\chapter{史瓦西黑洞物理}

\section{史瓦西黑洞性质}
史瓦西度规是一个具有对称性的时空,使用四维球坐标系$\left(ct,r,\theta,\phi\right)$.

史瓦西黑洞具有如下性质:
\begin{enumerate}
    
    \item what
    \item sdfsdf
\end{enumerate}


\section{史瓦西黑洞测地线与光线追踪方程的推导}

史瓦西黑洞测地线
\begin{equation}
    ds^{2}=c^{2}d\tau^{2}=c^{2}\left(1-\frac{2MG}{c^{2}r}\right)dt^{2}-\left(1-\frac{2MG}{c^{2}r}\right)^{-1}dr^{2}-r^{2}d\theta^{2}-r^{2}\sin\theta^{2}d\phi^{2}
\end{equation}
其中$M$是黑洞的质量, $G$是万有引力常数, $c$是真空光速.
对于无质量的粒子(光子)我们有 Null 测地线,
\begin{equation}
    g_{\mu\nu}u^{\mu}u^{\nu}=u^{\mu}u_{\mu}=0
\end{equation}
带入史瓦西黑洞测地线有

\begin{equation}
    \begin{split}
        0 & =-\left(1-\frac{2MG}{c^{2}r}\right)\left(c\frac{dt}{d\tau}\right)^{2}+\left(1-\frac{2MG}{c^{2}r}\right)^{-1}\left(\frac{dr}{d\tau}\right)^{2}\\
        & \qquad+r^{2}\left(\frac{d\theta}{d\tau}\right)^{2}+r^{2}\sin^{2}\theta\left(\frac{d\phi}{d\tau}\right)^{2}
    \end{split}
\end{equation}

因为史瓦西度规是一个球对称时空,粒子的运动可以简化成赤道面上的圆周运动. 令$\theta=\frac{\pi}{2}$, 则$d\theta=0$, 带入得,
\begin{equation}
    \begin{split}
        0&=-\left(1-\frac{2MG}{c^{2}r}\right)\left(c\frac{dt}{d\lambda}\right)^{2}+\left(1-\frac{2MG}{c^{2}r}\right)^{-1}\left(\frac{dr}{d\lambda}\right)^{2}+r^{2}\left(\frac{d\phi}{d\lambda}\right)^{2}
    \end{split}
\end{equation}

根据守恒性质,我们有两个Killing Vector
\begin{equation}
    \begin{split}
        \left(1-\frac{2GM}{c^{2}r}\right)c^{2}\frac{dt}{d\tau}&=\frac{E}{m_{0}}\\
        r^{2}\sin^{2}\theta\frac{d\phi}{d\tau}&=\frac{L}{m_{0}}
    \end{split}
\end{equation}
解得
\begin{equation}
    \begin{split}
        \frac{d\phi}{d\tau}&=\frac{L}{m_{0}r^{2}\sin^{2}\theta}\\
        c^{2}\left(\frac{dt}{d\tau}\right)^{2}&=\frac{E^{2}}{m_{0}^{2}c^{2}}\left(1-\frac{2GM}{c^{2}r}\right)^{-2}
    \end{split}
\end{equation}
将其带入,
\begin{equation}
    \begin{split}
        \frac{d\phi}{d\tau}&=\frac{L}{m_{0}r^{2}\sin^{2}\theta}\\
        c^{2}\left(\frac{dt}{d\tau}\right)^{2}&=\frac{E^{2}}{m_{0}^{2}c^{2}}\left(1-\frac{2GM}{c^{2}r}\right)^{-2}
    \end{split}
\end{equation}