%%
% The BIThesis Template for Bachelor Graduation Thesis
%
% 北京理工大学毕业设计(论文)结论 —— 使用 XeLaTeX 编译
%
% Copyright 2020 Spencer Woo
%
% This work may be distributed and/or modified under the
% conditions of the LaTeX Project Public License, either version 1.3
% of this license or (at your option) any later version.
% The latest version of this license is in
%   http://www.latex-project.org/lppl.txt
% and version 1.3 or later is part of all distributions of LaTeX
% version 2005/12/01 or later.
%
% This work has the LPPL maintenance status `maintained'.
%
% The Current Maintainer of this work is Spencer Woo.
%
% Compile with: xelatex -> biber -> xelatex -> xelatex

\addcontentsline{toc}{chapter}{结~~~~论}
\chapter*{\vskip 10bp\textmd{结~~~~论} \vskip -6bp}

% 在结论部分的子标题不需要序号,加上 * 即可(一个例子如下)
% \section*{结论段落标题}

% 这里插入一个参考文献,仅作参考
本文的主要目的是实现一个具象化的黑洞图像,让人在视觉上有一个直观的认识。所做成的软件可以作为应用软件在普通家用电脑上运行。渲染器本身还可以添加更多的功能,比如说可以让吸积盘拥有厚度、透明度或者直接用动态粒子来实现吸积盘。

渲染器还可以实现更多的黑洞种类,如自转的Kerr黑洞,会形成不是圆形的爱因斯坦环。软件的渲染速度还可以优化,包括吸积盘的采样过程、对单调函数的积分方法优化等。
